\newpage
\section{Programming Guide}
\label{sec:programming}

Performant vector code on Saturn is generally straightforward to implement and requires less microarchitectural fine-tuning than other vector or SIMD systems.
Still, some general guidelines for vector programming apply.
This section is divided into discussions on microarchitecturally-relevant areas of interest when writing vector code.


Saturn's implementation is built around the machine datapath width, \texttt{DLEN}.
The fundamental occupancy of one vector instruction, or chime length, in a given Saturn implementation is \texttt{VLEN/DLEN}.
In most implementations, this will be 2, although Saturn can generate implementations with a chime length of 1 or greater.
Using the \texttt{LMUL} parameter allows the programmer to extend the chime length to \texttt{LMUL x VLEN/DLEN}, which increases the effective occupancy of each vector instruction.

Generally speaking, efficient vector code should maximize the utilization of the functional units.
Saturn efficiently implements vector register grouping, and thus programmers should generally use the largest possible \texttt{LMUL} that applications allow without causing excessive vector register spilling.
Maximizing \texttt{LMUL} decreases dynamic instruction count and increases chime length, avoiding underutilization due to insufficient instruction throughput.

\subsection{Optimizing Towards Chaining}

Leveraging vector chaining is critical for achieving high utilization of the functional units.
Saturn implements vector chaining at \texttt{DLEN}, or element-group granularity, through the vector register file (direct functional unit to functional unit bypasses are not implemented).
Chaining enables a dependent vector instruction to begin sequencing operations as soon as its parent instruction writes back its first element group, assuming no other structural or data hazards.

Chaining only occurs between instructions occupying different sequencers, as the sequencers are single-issue in-order.
All sequencers can chain between each other.
For example, arithmetic operations may chain off of memory operations and vice-versa.
When using a configuration with multiple execute sequencers, independent arithmetic operations may chain off each other and overlap execution.
Optimized vector code should seek to load-balance instructions across the sequencers such that chaining might occur.


\subsection{Optimizing Around Issue Limitations}

To maximize the utilization of all functional units, the issue structure of the machine should be considered.
The issue queue structure and depths are parameterized in Saturn and may differ across different configurations.

In the ``Unified" structure, all vector arithmetic instructions, whether integer or floating-point, execute from the same sequencer, so no parallel arithmetic execution is possible.
In the ``Shared" and ``Split" structures, independent sequencers drive the integer and floating-point execution units, so parallel execution is possible under certain conditions.
In the ``Shared" configuration, since the sequencers share a single in-order issue queue, interleaving vector instructions across the integer and floating-point paths is critical.
In the ``Split" configuration, each sequencer has its own issue queue, reducing the importance of software-scheduled load balancing.

When the issue queue sizes are shallow, the issue queues may fill up and back-pressure the decoupling queue, and consequently, the scalar core's frontend.
Even modest configurations of Saturn should contain enough buffering within the issue and decoupling queues to avoid back-pressure for well-load-balanced code.


\subsection{Optimizing Around the Scalar Core}

Saturn is designed to be performant even when integrated with a compact, single-issue, in-order scalar core.
Specific details related to Rocket and Shuttle can be found in Section \ref{sec:frontend}.
When writing loops, scalar bookkeeping instructions should be scheduled to overlap with the execution of the vector instructions.

Saturn configurations with short chime lengths are preferably integrated with Shuttle, as short-chime instructions generally require higher frontend instruction throughput.
When operating on long vector lengths with long chimes, the performance difference between Rocket and Shuttle diminishes.
As discussed in Section \ref{sec:frontend}, the handling of \texttt{vset} instructions varies across different scalar cores.
This may induce pipeline bubbles and should be considered when writing kernels.  

Due to the post-commit execution of vector instructions, vector instructions that write scalar registers should be used sparingly.
As the supported scalar cores do not implement speculative execution, RAW dependencies on vector-produced scalar results will cause the scalar pipeline to stall.
These should be avoided in inner loops.


\subsection{Optimizing Around Pipeline Latencies}

A single vector instruction requires \texttt{VLEN/DLEN} (the chime length) cycles of occupancy on most \texttt{DLEN}-wide functional unit datapaths.
When this chime length is less than the pipeline depth of the relevant functional unit, the proceeding instruction will stall due to a data dependency on the yet-to-be-produced writeback data.
This situation is rare due to the support for chaining, but might still appear in a sequence of low-\texttt{LMUL} floating point instructions.

To saturate the FMA units in this scenario, either a longer \texttt{LMUL} should be used, or independent FMAs must be scheduled back-to-back.
Generally, performant code should use the highest \texttt{LMUL} possible that avoids vector register spilling.


\subsection{Optimizing Segmented Memory Accesses}

Saturn's implementation strives to implement complex instructions efficiently.
In particular, the segmented loads and stores that reformat memory into vector registers are implemented to be able to fully utilize the wide memory port.
As a general rule, using such complex instructions should never induce a performance penalty over an equivalent longer sequence of simpler instructions.
Thus, programmers should use complex instructions whenever possible, even if the kernel could be equivalently implemented by a longer sequence of simpler instructions.


\subsection{Optimizing Reductions}

Currently, Saturn cannot maintain full utilization for element-wise reductions due to an insufficient number of accumulator registers to track intermediate values.
Classic techniques for vectorizing reduction operations should be utilized where possible. 
For example, applicable reduction loops should be expressed as a sequence of element-wise operations before a final reduction across all elements in the tail code.



